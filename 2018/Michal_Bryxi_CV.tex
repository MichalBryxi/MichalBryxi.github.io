% !TEX program = lualatex
%%%%%%%%%%%%%%%%%
% This is an example CV created using altacv.cls (v1.1.4, 27 July 2018) written by
% LianTze Lim (liantze@gmail.com), based on the
% Cv created by BusinessInsider at http://www.businessinsider.my/a-sample-resume-for-marissa-mayer-2016-7/?r=US&IR=T
%
%% It may be distributed and/or modified under the
%% conditions of the LaTeX Project Public License, either version 1.3
%% of this license or (at your option) any later version.
%% The latest version of this license is in
%%    http://www.latex-project.org/lppl.txt
%% and version 1.3 or later is part of all distributions of LaTeX
%% version 2003/12/01 or later.
%%%%%%%%%%%%%%%%

%% If you want to use \orcid or the
%% academicons icons, add "academicons"
%% to the \documentclass options.
%% Then compile with XeLaTeX or LuaLaTeX.
% \documentclass[10pt,a4paper,academicons]{altacv}

%% Use the "normalphoto" option if you want a normal photo instead of cropped to a circle
% \documentclass[10pt,a4paper,normalphoto]{altacv}

\documentclass[10pt,a4paper]{altacv}

%% AltaCV uses the fontawesome and academicon fonts
%% and packages.
%% See texdoc.net/pkg/fontawecome and http://texdoc.net/pkg/academicons for full list of symbols.
%% When using the "academicons" option,
%% Compile with LuaLaTeX for best results. If you
%% want to use XeLaTeX, you may need to install
%% Academicons.ttf in your operating system's font %% folder.


% Change the page layout if you need to
\geometry{left=1cm,right=9cm,marginparwidth=6.8cm,marginparsep=1.2cm,top=1cm,bottom=1cm}

% Change the font if you want to.

% If using pdflatex:
\usepackage[utf8]{inputenc}
\usepackage[T1]{fontenc}
\usepackage[default]{lato}

% If using xelatex or lualatex:
\setmainfont{Lato}

% Change the colours if you want to
\definecolor{VividPurple}{HTML}{3E0097}
\definecolor{SlateGrey}{HTML}{2E2E2E}
\definecolor{LightGrey}{HTML}{666666}
\colorlet{heading}{VividPurple}
\colorlet{accent}{VividPurple}
\colorlet{emphasis}{SlateGrey}
\colorlet{body}{LightGrey}

% Change the bullets for itemize and rating marker
% for \cvskill if you want to
\renewcommand{\itemmarker}{{\small\textbullet}}
\renewcommand{\ratingmarker}{\faCircle}

%% sample.bib contains your publications
\addbibresource{sample.bib}

\begin{document}
\name{Michal Bryxí}
\tagline{Full stack developer}
% Cropped to square from https://en.wikipedia.org/wiki/Marissa_Mayer#/media/File:Marissa_Mayer_May_2014_(cropped).jpg, CC-BY 2.0
\photo{2.5cm}{picture.png}
\personalinfo{%
  % Not all of these are required!
  % You can add your own with \printinfo{symbol}{detail}
  \email{michal.bryxi@gmail.com}
  \phone{0044 747 534 6614}
  % \mailaddress{Address, Street, 00000 County}
  \location{Belfast, Northern Ireland}
  \homepage{https://pudr.com}
  \twitter{@MichalBryxi}
  \linkedin{linkedin.com/in/michalbryxi}
  % \printinfo{\faGitlab}{bar}
  % \github{github.com/mmayer} % I'm just making this up though.
%   \orcid{orcid.org/0000-0000-0000-0000} % Obviously making this up too. If you want to use this field (and also other academicons symbols), add "academicons" option to \documentclass{altacv}
}

%% Make the header extend all the way to the right, if you want.
\begin{fullwidth}
\makecvheader
\end{fullwidth}

%% Depending on your tastes, you may want to make fonts of itemize environments slightly smaller
\AtBeginEnvironment{itemize}{\small}

%% Provide the file name containing the sidebar contents as an optional parameter to \cvsection.
%% You can always just use \marginpar{...} if you do
%% not need to align the top of the contents to any
%% \cvsection title in the "main" bar.
\cvsection[page1sidebar]{Experience}

\cvevent{Front-end developer}{Puppet}{2013 -- 2018}{Belfast, Northern Ireland}
\begin{itemize}
  \item Development of ambitious Ember.js applications - Puppet Enterprise Console
\end{itemize}

\divider

\cvevent{Configuration manager}{IntraWorlds s.r.o.}{2010 -- 2013}{Pilsen, Czechia}
\begin{itemize}
  \item Linux infrastructure maintenance and growth. OS/application security and performance. Automation of daily developer tasks.
  \item Product migration from ISO 8859-1 to UTF8.
  \item Implementation of ISO 27001 - information security management system.
  \item Infrastructure migration to Puppet - IT automation software.
  \item Implementation of central logging tool - Logstash.
\end{itemize}

\divider

\cvevent{Configuration manager}{vsechnyzakazky.cz}{2012}{}
\begin{itemize}
  \item{Linux server administration - puppet; virtualization - libvirt and KVM; security}
\end{itemize}

\divider

\cvevent{Webmaster, linux administrator}{dione.zcu.cz}{2008 -- 2011}{}

\begin{itemize}
  \item Implementation of e-zine based on Drupal CMS.
  \item Linux system administration.
\end{itemize}

\divider

\cvevent{Lector on network administrator course}{Jan Ámos Komenský academy}{2008}{}

\begin{itemize}
  \item{Network administrator course for employment office.}
\end{itemize}

% \divider

% \cvevent{Product Engineer}{Google}{23 June 1999 -- 2001}{Palo Alto, CA}

% \begin{itemize}
% \item Joined the company as employe \#20 and female employee \#1
% \item Developed targeted advertisement in order to use user's search queries and show them related ads
% \end{itemize}

\cvsection{My free time}

% Adapted from @Jake's answer from http://tex.stackexchange.com/a/82729/226
% \wheelchart{outer radius}{inner radius}{
% comma-separated list of value/text width/color/detail}
% Some ad-hoc tweaking to adjust the labels so that they don't overlap
\wheelchart{1.5cm}{0.5cm}{%
  10/10em/accent!30/Running,
  25/9em/accent!60/Rock climbing,
  5/13em/accent!10/Side projects \footnotesize(mostly IT related),
  20/15em/accent!40/Cycling,
  5/8em/accent!20/Traveling,
  30/9em/accent/Geocaching,
  5/8em/accent!20/
}

\clearpage

\cvsection[page2sidebar]{Publications}

\nocite{*}

\printbibliography[heading=pubtype,title={\printinfo{\faBook}{Books}},type=book]

\divider

\printbibliography[heading=pubtype,title={\printinfo{\faFileTextO}{Journal Articles}}, type=article]

\divider

\printbibliography[heading=pubtype,title={\printinfo{\faGroup}{Conference Proceedings}},type=inproceedings]

%% If the NEXT page doesn't start with a \cvsection but you'd
%% still like to add a sidebar, then use this command on THIS
%% page to add it. The optional argument lets you pull up the
%% sidebar a bit so that it looks aligned with the top of the
%% main column.
% \addnextpagesidebar[-1ex]{page3sidebar}


\end{document}
